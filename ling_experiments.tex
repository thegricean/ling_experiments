\documentclass[11pt]{article}

\usepackage{apacite}
\usepackage{amsmath,amssymb}
\usepackage{graphicx}
\usepackage{color}
\usepackage{url}
\usepackage{fullpage}
\usepackage{setspace}
\usepackage{booktabs}
\usepackage{multirow}
\usepackage{lingmacros}
\usepackage{caption}
\usepackage{subcaption}
\usepackage{tablefootnote}
\usepackage{hyperref}

\definecolor{Red}{RGB}{255,0,0}
\newcommand{\red}[1]{\textcolor{Red}{#1}}
\newcommand{\jd}[1]{\textcolor{Red}{[jd: #1]}} 

\newcommand{\subsubsubsection}[1]{{\em #1}}
\newcommand{\eref}[1]{(\ref{#1})}
\newcommand{\tableref}[1]{Table \ref{#1}}
\newcommand{\figref}[1]{Figure \ref{#1}}
\newcommand{\appref}[1]{Appendix \ref{#1}}
\newcommand{\sectionref}[1]{Section \ref{#1}}

\hypersetup{colorlinks,breaklinks,
            linkcolor=cyan,urlcolor=cyan,
            anchorcolor=cyan,citecolor=cyan}

\title{If you're a linguist and want to conduct an experiment, clap your hands}

 
\author{{\large \bf Judith Degen and Leon Bergen (or the other way around)} \\
  \{jdegen,bergen\}@stanford.edu\\
  Department of Psychology, 450 Serra Mall \\
  Stanford, CA 94305 USA}

\begin{document}

\maketitle


\begin{abstract}
Linguistics is taking an experimental turn. This is a great development in principle -- it means new sources of data for linguistic theories to draw upon and reflects linguists' renewed willingness to engage with the conception of language as action produced by minds. At the same time, we argue, the experimental turn is a disastrous development. At present, very few linguists are trained in experimental design. However, the sentiment in the field appears to be that anyone can conduct an experiment. The result is that a great amount of experimental work of very little merit makes its way into peer review, if not beyond. We argue that this is at best a huge waste of time for those of us who have to review these papers. At worst, the field is  becoming flooded with poorly controlled, poorly analyzed, and heavily confounded work. Here we provide the Experimentally Mature Linguist (EML) Test, which consists of a series of XXX flawed studies. Only those who can find the flaw and suggest an improvement should even consider running an experiment. For the others we provide a list of resources. Seriously, if you want to run experiments you should feel the terror of being cast into the \href{https://www.youtube.com/watch?v=cV0tCphFMr8}{Gorge of Peril} about getting one of these wrong. 


\textbf{Keywords:} 
linguistics; experiments
\end{abstract}

\section{Introduction}
\label{sec:intro}

\jd{instill fear}

\section{The EML Bridge of Death}
\label{sec:bod}

\jd{What is your name? What is your quest? What is the airspeed velocity of an unladen swallow?}

\section{Resources}

\jd{books, papers, courses, labs, and gods we trust}

%\bibliographystyle{apacite}
%
%\setlength{\bibleftmargin}{.125in}
%\setlength{\bibindent}{-\bibleftmargin}
%
%\bibliography{bibs}


\end{document}
